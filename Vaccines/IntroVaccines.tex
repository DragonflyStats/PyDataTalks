
Course Schedule:

- Week 1 – October 29th:

Course Overview (Karen Charron and Amber Cox)

Lecture 1: Type of Vaccines, How Vaccines are Administered, and the Vaccine Development Process (Neal Halsey)

Lecture 2: Essential Elements of a Vaccine Protocol (Karen Charron)

- Week 2 – November 4th:

Lecture 3: Characteristics and Outcomes of Vaccine Trials (Clayton Harro)

Lecture 4: Protection of Human Subjects (Amber Cox)

- Week 3 – November 11th:

Lecture 5: Vaccine Trials in Pediatric Populations (Elizabeth Schappell)

Lecture 6: Vaccine Management and Preparation (Hye Kim and Vivian Rexroad)

- Week 4 – November 18th:

Lecture 7: Data Management, Quality Assurance, and Quality Control (Karen Charron)

- Week 5 – November 25th:

Lecture 8: Community Involvement and Recruitment (James Williams)

Lecture 9: Study Start-Up and Implementation (Karen Charron)

- Week 6 – December 2nd:

Lecture 10: Screening Volunteers (Karen Charron)

Lecture 11: Enrollment and Implementation Procedures (Karen Charron)

- Week 7 – December 9th:

Lecture 12: Safety Assessments and Management of Adverse Events (Anna Durbin)

Course Resources:

There are no required textbooks or readings for this course, but the following resources may be useful.
•
FDA Good Clinical Practice 2011 Reference Guide (Book 1B). This book contains FDA Good Clinical Practice regulations (21 CFR Parts 11, 50, 54, 56, and 312) and ICH Guidelines (E6, E2A, and E8), as well as the FDA Information Sheets and other guidance documents that govern the conduct of clinical trials for drug research. It can be ordered online for $14.95. http://www.clinicalresearchresources.com/books/bookstore.html

•
E6 Good Clinical Practice Consolidated Guidance. This PDF file is available from the F.D.A. website.http://www.fda.gov/downloads/Drugs/GuidanceComplianceRegulatoryInformation/Guidances/UCM073122.pdf

•
ClinicalTrials.gov. This website is maintained by the U.S. NIH and is a database providing information on both publicly and privately funded clinical studies enrolling volunteers worldwide. http://clinicaltrials.gov/ct2/home


We will also post important reference documents in the Online Library, including:
•
Declaration of Helsinki

•
Belmont Report

•
Beecher paper

•
Forms 1571 and 1572


Grading Policy:
•
You must complete all 12 quizzes with a score of 80% or better to successfully complete this class. (You will have the opportunity to take each quiz up to 10 times to improve your score.)

•
We recommend taking each quiz immediately after viewing the lecture material, but the quizzes will remain open for the duration of the course.

•
Quizzes must be completed by 11:59 pm EST on Sunday, December 16th in order to receive credit.



