Population health management (PHM)
 
One method to improve population health is population health management (PHM), which has been defined as “the technical field of endeavor which utilizes a variety of individual, organizational and cultural interventions to help improve the morbidity patterns (i.e., the illness and injury burden) and the health care use behavior of defined populations”.[14] PHM is distinguished from disease management by including more chronic conditions and diseases, by use of "a single point of contact and coordination", and by "predictive modeling across multiple clinical conditions".[15] PHM is considered broader than disease management in that it also includes "intensive care management for individuals at the highest level of risk" and "personal health management... for those at lower levels of predicted health risk".[16] Many PHM-related articles are published in Population Health Management, the official journal of DMAA: The Care Continuum Alliance.[17]
 
The following road map has been suggested for helping healthcare organizations navigate the path toward implementing effective population health management:[18]
 Establish precise patient registries
 Determine patient-provider attribution
 Define precise numerators in the patient registries
 Monitor and measure clinical and cost metrics
 Adhere to basic clinical practice guidelines
 Engage in risk-management outreach
 Acquire external data
 Communicate with patients
 Educate patients and engage with them
 Establish and adhere to complex clinical practice guidelines
 Coordinate effectively between care team and patient
 Track specific outcomes
